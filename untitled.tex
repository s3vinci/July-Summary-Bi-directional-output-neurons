\section{ABSTRACT}


The fruit fly uses its sense of smell to discover useful facts about the world. \iffalse Strange odours are populating the world everywhere, every being tries to smell it as the molecules propagate without end \fi  Previous experiences of different smells get stored in
the fly brain along with information on whether the smell was associated
with a reward or punishment. \iffalse Reward or punishment, as if this is the be all end all of everything. It's not we don't think in binary, things don't exist in only two states above the scale of quanta. \fi Flies will use their sense of smell to
decide whether to approach or retreat from a given odour. The place
this information gets stored is the mushroom body, an area comprised
of thousands of neurons that has been implicated in olfactory memory
and decision-making. Experiments have shown the response of a fly
to a given odour is determined by the activity of a small group of
so called \textquotedblleft output neurons\textquotedblright , which
receive their input from the mushroom body and are also affected by
neuromodulators such as dopamine and octopamine, with experimental
evidence suggesting that dopamine and octopamine can control reward
based learning. Silencing or activating this group of neurons can
artificially change the behaviour of the fly from approach to avoidance.
In this study I will determine how these neuromodulators interact
with synaptic plasticity. We built a rate model of the olfactory circuit to test the interaction between dopamine to determine which plasticity mechanisms are sufficient to store information related to reward associated with odour stimuli between the mushroom body and output neurons.

In this short project we have studied the ability of the fruit fly's olfactory circuit to learn how to select an appropriate action from a behavioural repertoire when exposed to an odour it has previously learned through first order conditioning. Previous research has uncovered the role of three separate groups. 

The studies of Laurent and co-authors on the olfactory processing in Locusts \cite{25864636}   emphasise the role of time in olfactory conditioning. 
  
  
  
  
  
  
  
 
  
  
  
  